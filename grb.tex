%% Copyright 2007-2020 Elsevier Ltd
%% 
%% This file is part of the 'Elsarticle Bundle'.
%% ---------------------------------------------
%% 
%% It may be distributed under the conditions of the LaTeX Project Public
%% License, either version 1.2 of this license or (at your option) any
%% later version.  The latest version of this license is in
%%    http://www.latex-project.org/lppl.txt
%% and version 1.2 or later is part of all distributions of LaTeX
%% version 1999/12/01 or later.
%% 
%% The list of all files belonging to the 'Elsarticle Bundle' is
%% given in the file `manifest.txt'.
%%
%% Template article for Elsevier's document class `elsarticle'
%% with harvard style bibliographic references
%% 
% \documentclass[preprint,12pt,authoryear]{elsarticle}
%%
%% Use the option review to obtain double line spacing
%% \documentclass[authoryear,preprint,review,12pt]{elsarticle}
%%
%% Use the options 1p,twocolumn; 3p; 3p,twocolumn; 5p; or 5p,twocolumn
%% for a journal layout:
%% \documentclass[final,1p,times,authoryear]{elsarticle}
%% \documentclass[final,1p,times,twocolumn,authoryear]{elsarticle}
%% \documentclass[final,3p,times,authoryear]{elsarticle}
%% \documentclass[final,3p,times,twocolumn,authoryear]{elsarticle}
%% \documentclass[final,5p,times,authoryear]{elsarticle}
\documentclass[final,5p,times,twocolumn,authoryear]{elsarticle}

%--- idioma -----------------------------------------------------------
\usepackage[spanish, es-noshorthands]{babel}          % opcional: traducir los encabezados
\usepackage[utf8]{inputenc}
\usepackage[T1]{fontenc}
\usepackage{amsmath}
% ...tus otros paquetes...
% \usepackage[hyperlinks]{hyperref} % sin color ni recuadros
\usepackage[]{hyperref} % a mi sí me gustan los colores :D -Aless
\hypersetup{
    colorlinks=true,
    linkcolor=blue,
    filecolor=magenta,      
    urlcolor=blue,
    }
\urlstyle{same}                  % URLs con la misma tipografía del texto
\usepackage{subcaption}


%--- redefinir los encabezados ----------------------------------------
\abstracttitle{Resumen}
\keywordtitle{Palabras clave}

%% For including figures, graphicx.sty has been loaded in
%% elsarticle.cls. If you prefer to use the old commands
%% please give \usepackage{epsfig}

%% The amssymb package provides various useful mathematical symbols
\usepackage{amssymb}
% \usepackage{lipsum}
%% The amsthm package provides extended theorem environments
%% \usepackage{amsthm}

%% The lineno packages adds line numbers. Start line numbering with
%% \begin{linenumbers}, end it with \end{linenumbers}. Or switch it on
%% for the whole article with \linenumbers.
%% \usepackage{lineno}

%% You might want to define your own abbreviated commands for common used terms, e.g.:
\newcommand{\kms}{km\,s$^{-1}$}
\newcommand{\msun}{$M_\odot$}

\journal{Astronomy $\&$ Computing}

\setlength{\parskip}{3mm}

\begin{document}

\begin{frontmatter}

%% Title, authors and addresses

%% use the tnoteref command within \title for footnotes;
%% use the tnotetext command for theassociated footnote;
%% use the fnref command within \author or \affiliation for footnotes;
%% use the fntext command for theassociated footnote;
%% use the corref command within \author for corresponding author footnotes;
%% use the cortext command for theassociated footnote;
%% use the ead command for the email address,
%% and the form \ead[url] for the home page:
%% \title{Title\tnoteref{label1}}
%% \tnotetext[label1]{}
%% \author{Name\corref{cor1}\fnref{label2}}
%% \ead{email address}
%% \ead[url]{home page}
%% \fntext[label2]{}
%% \cortext[cor1]{}
%% \affiliation{organization={},
%%            addressline={}, 
%%            city={},
%%            postcode={}, 
%%            state={},
%%            country={}}
%% \fntext[label3]{}

\title{Análisis bn090424592}

%% use optional labels to link authors explicitly to addresses:
%% \author[label1,label2]{}
%% \affiliation[label1]{organization={},
%%             addressline={},
%%             city={},
%%             postcode={},
%%             state={},
%%             country={}}
%%
%% \affiliation[l\abel2]{organization={},
%%             addressline={},
%%             city={},
%%             postcode={},
%%             state={},
%%             country={}}

\author[ecfm]{Alessandro Lavagnino}
\author[ecfm]{Brian Leiva}
\author[ecfm]{Francisco Toledo}
\author[ecfm]{Andy Pérez}

\affiliation[ecfm]{
  organization={ECFM, Universidad de San Carlos de Guatemala}
  % city={Guatemala},
  % country={Guatemala.}
}
% \author[first]{Alessandro Lavagnino}
% \affiliation[third]{organization={ECFM, USAC},%Department and Organization
%             % addressline={}, 
%             city={Guatemala},
%             % postcode={}, 
%             % state={Guatemala},
%             country={Guatemala}}

% \author[second]{Brian Leiva}
% \affiliation[second]{organization={ECFM, Universidad de San Carlos de Guatemala},%Department and Organization
%             % city={Guatemala},
%             % state={Guatemala},
%             country={Guatemala}}

% \author[third]{Author name}
% \affiliation[first]{organization={ECFM, Universidad de San Carlos de Guatemala},%Department and Organization
%             addressline={}, 
%             city={Guatemala},
%             postcode={}, 
%             state={Guatemala},
%             country={Guatemala}}

\begin{abstract}
%% Text of abstract
Example abstract for the astronomy and computing journal. Here you provide a brief summary of the research and the results.
\end{abstract}

%%Graphical abstract
%\begin{graphicalabstract}
%\includegraphics{grabs}
%\end{graphicalabstract}

%%Research highlights
%\begin{highlights}
%\item Research highlight 1
%\item Research highlight 2
%\end{highlights}

\begin{keyword}
%% keywords here, in the form: keyword \sep keyword, up to a maximum of 6 keywords
estallido de rayos gamma: individual: GRB 090926A \sep
Astrofísica - Fenómenos astrofísicos de alta energía \sep
Astrofísica - Cosmología y astrofísica extragaláctica



%% PACS codes here, in the form: \PACS code \sep code

%% MSC codes here, in the form: \MSC code \sep code
%% or \MSC[2008] code \sep code (2000 is the default)

\end{keyword}


\end{frontmatter}

%\tableofcontents

%% \linenumbers

%% main text

\section{Introducción}
\label{introduction}

Los destellos de rayos gamma (GRB) son explosiones transitorias de muy alta
energía que liberan, en cuestión de segundos, una cantidad de energía comparable
a la de una galaxia entera \citep{carroll2017}. Su emisión se divide en dos
fases observacionales.

En primer lugar, la emisión principal (prompt) constituye la fase inicial, cuya
duración varía desde milisegundos hasta varios cientos de segundos.  Durante
este intervalo se detecta radiación de alta energía (keV-GeV) caracterizada por
flujos de fotones extremadamente rápidos y variaciones temporales muy rápidas.

Posteriormente, la emisión tardía (afterglow) se manifiesta cuando el choque
relativista interactúa con el medio circundante, produciendo una radiación más
prolongada que se extiende desde los rayos X hasta el rango radiofónico.

% Comprender este doble comportamiento es esencial para identificar los
% mecanismos de producción de GRB, que pueden originarse tanto por el
% colapso de estrellas masivas como por la fusión de objetos compactos.

\section{Antecedentes}
\label{antecedentes}
% Sugerencias antecedentes:
% Ajuste Band
% 

%%\subsection{Historia}

% Los estallidos de rayos gamma (GRB) se identificaron por primera vez de manera
% fortuita a finales de la década de 1960, en el contexto de la Guerra Fría. Para
% verificar el cumplimiento del tratado que prohibía ensayos nucleares en el
% espacio, Estados Unidos lanzó los satélites \textit{Vela}, equipados con
% detectores de rayos X y gamma, lo que permitió registrar señales de alta energía
% no atribuibles a detonaciones nucleares \citep{bonnell1995}. El 2 de julio de
% 1967, dos de estos satélites (Vela~3 y Vela~4) detectaron un destello de rayos
% gamma sin correspondencia con ninguna firma conocida; el equipo de Los Alamos,
% encabezado por Ray Klebesadel, archivó el evento para un análisis posterior
% \citep{bonnell1995}.

% Con el despliegue de versiones más avanzadas de los \textit{Vela} en los años
% siguientes, se detectaron múltiples destellos similares. A partir de las
% pequeñas diferencias en los tiempos de llegada entre naves separadas, el grupo
% estimó posiciones aproximadas de 16 eventos y demostró que su origen no era
% terrestre ni solar, sino que apuntaba al espacio profundo
% \citep{bonnell1995,klebesadel1973}. El hallazgo se comunicó públicamente en 1973
% mediante el artículo \textit{Observations of Gamma-Ray Bursts of Cosmic Origin}
% en \textit{Astrophysical Journal Letters}, donde se reportó por primera vez la
% existencia de fuentes transitorias de rayos gamma de origen cósmico,
% estableciendo un nuevo campo de investigación que impulsaría la astrofísica de
% altas energías durante las décadas siguientes \citep{klebesadel1973}.

% \subsection{Carecterización}

% Un GRB típico se presenta como un destello súbito e intensísimo de rayos gamma
% ---la forma más energética de la radiación electromagnética--- procedente de una
% localización remota del universo. Su duración varía enormemente: puede ir de
% fracciones de milisegundo a varios minutos, aunque la mayoría apenas se extiende
% por unos pocos segundos. Inmediatamente después del pulso gamma inicial, casi
% todos los GRB exhiben una post-luminiscencia (\textit{afterglow}) a longitudes
% de onda mayores: emisión en rayos X, ultravioleta, luz visible, infrarrojo e
% incluso radio, que puede persistir durante días, semanas o más conforme la
% explosión interactúa con el medio interestelar circundante. El descubrimiento
% del \textit{afterglow} en 1997, gracias al satélite italo-holandés
% \textit{BeppoSAX}, fue un hito que permitió localizar por primera vez la galaxia
% anfitriona de un GRB y medir su distancia cosmológica mediante el corrimiento al
% rojo. Esto confirmó que los GRB ocurren a miles de millones de años luz de la
% Tierra y liberan energías colosales: en unos pocos segundos un GRB puede emitir
% tanta energía como el Sol a lo largo de toda su vida (\(\sim 10^{10}\) años).
% Afortunadamente, pese a su potencia descomunal, estos eventos son
% extraordinariamente infrecuentes en una galaxia dada (quizá del orden de uno por
% millón de años por galaxia), de modo que es muy improbable que sucedan en la Vía
% Láctea en tiempos cercanos. Todos los GRB observados hasta hoy han sido
% extragalácticos y su distribución aparentemente isotrópica en el cielo respalda
% su origen a distancias cosmológicas, no dentro de nuestra galaxia. En casos
% raros se han detectado fenómenos análogos más débiles en la Vía Láctea,
% asociados a magnetares (estallidos de rayos gamma suaves), pero los GRB
% ``clásicos'' provienen de objetos situados en el universo distante
% \citep{Xu2009GRB}.

%\subsection{Origen}
Existen dos fuentes conocidas de GRBs, el colapso de estrellas masivas y la fusión de objetos compactos
\vspace{0.8\baselineskip} % ajusta 0.3–1.0 a tu gusto

\subsection{Colapso de estrellas masivas}
Cuando una estrella muy masiva ($\sim 20 M_{\odot}$ o más) agota su combustible
nuclear, su núcleo colapsa rápidamente formando un agujero negro o una
estrella de neutrones. El colapso genera una hipernova y produce chorros
relativistas que perforan la envoltura estelar, dando lugar a los GRB de
larga duración cuya emisión gamma se extiende durante varios segundos o
minutos.
\subsection{Fusión de objetos compactos}
En sistemas binarios compuestos por dos estrellas de neutrones o por una
estrella de neutrones y un agujero negro, la pérdida de energía por ondas
gravitacionales hace que las órbitas se contraigan hasta la colisión.
La coalescencia libera una gran cantidad de energía en menos de $2$ s,
generando chorros relativistas que producen los GRB de corta duración.

% \subsection{Detección y observación}

% Tras la publicación del descubrimiento de los GRB en 1973, numerosos
% experimentos espaciales se volcaron en acumular más eventos y desentrañar su
% origen. Un hito clave llegó entre 1991 y 2000 con el instrumento BATSE (Burst
% and Transient Source Experiment), a bordo del Observatorio de Rayos Gamma
% Compton de la NASA. Durante sus nueve años de operación, BATSE registró 2 704
% estallidos en todo el cielo y proporcionó la primera distribución estadística
% verdaderamente significativa de GRB. De forma sorprendente, esa distribución
% mostró que los estallidos llegan desde todas las direcciones con una casi
% perfecta isotropía, sin preferencia por el plano de la Vía Láctea. La ausencia
% de concentración galáctica, combinada con un déficit de GRB débiles (indicativo
% de un horizonte de detección), constituyó una evidencia contundente de su origen
% a distancias cosmológicas, más allá de nuestra galaxia. Aun así, durante buena
% parte de las décadas de 1980 y 1990 permaneció abierta la incógnita sobre qué
% fenómenos astrofísicos podían producir estallidos tan remotos y energéticos.\\

% El siguiente gran hito observacional fue logrado en febrero de 1997 por el
% satélite BeppoSAX, cuando detectó por primera vez la post-luminiscencia en rayos
% X de un GRB (evento GRB 970228) y rápidamente telescopios en tierra captaron su
% débil resplandor en luz visible. Esto permitió identificar la galaxia anfitriona
% y medir un corrimiento al rojo de \(z \approx 0.695\), confirmando sin lugar a
% dudas que los GRB provenían de miles de millones de años luz de distancia. A
% partir de ese momento, el ``cortafuegos'' que impedía investigar la naturaleza
% de los GRB (su localización imprecisa) desapareció, y se sucedieron
% descubrimientos que conectaron a los GRB largos con supernovas (por ejemplo, GRB
% 980425 con SN 1998bw) y a los GRB cortos con galaxias elípticas envejecidas
% pobres en formación estelar (sugiriendo fusiones de objetos compactos).

% En la actualidad, la detección de GRB se realiza de forma rutinaria con
% observatorios espaciales dedicados. El Neil Gehrels \textit{Swift} Observatory
% (misión \textit{Swift} de la NASA, lanzada en 2004) marcó un parteaguas al
% incorporar un sistema de alerta y seguimiento automático. \textit{Swift}
% incorpora un detector de campo amplio de rayos gamma, el \textit{Burst Alert
% Telescope} (BAT), que descubre en promedio \(\sim 100\) GRB por año y determina
% su posición en el cielo con una precisión de pocos minutos de arco en cuestión
% de segundos. \citep{NASA_Swift_About} Acto seguido, la nave se reorienta de
% manera autónoma en menos de \(\sim 90\) segundos para apuntar sus telescopios de
% rayos X (XRT) y ultravioleta/óptico (UVOT) hacia la fuente y observar el
% \textit{afterglow} desde sus fases iniciales. \citep{NASA_Swift_About} Gracias a
% esta estrategia ``rápida'' (de ahí el nombre \textit{Swift}), por primera vez
% fue posible registrar la emisión remanente de numerosos GRB en múltiples
% longitudes de onda casi en tiempo real, obteniendo curvas de luz y espectros
% detallados que permiten inferir propiedades físicas del estallido (energía,
% colimación, entorno circundante, etc.). \textit{Swift} ha aportado miles de
% detecciones y un flujo de datos abierto a la comunidad, consolidándose hasta hoy
% como uno de los pilares en la investigación de estos eventos.\\

% Otra misión fundamental es el Telescopio Espacial de Rayos Gamma Fermi (NASA,
% lanzado en 2008), dotado de dos instrumentos complementarios: el Gamma-Ray Burst
% Monitor (GBM), sensible a rayos X/\(\gamma\) entre \(8~\mathrm{keV}\) y
% \(40~\mathrm{MeV}\), y el Large Area Telescope (LAT), que detecta fotones de
% energías más altas (desde \(\sim 20~\mathrm{MeV}\) hasta \(>100~\mathrm{GeV}\)).
% El GBM registra en torno a \(\sim 240\) GRB al año, cubriendo prácticamente todo
% el cielo no oculto por la Tierra, lo que lo convierte en el instrumento con
% mayor número de detecciones. El LAT, por su parte, observa los GRB más
% energéticos (\(\approx 10\) eventos anuales por encima de \(30~\mathrm{MeV}\)),
% haciendo posible estudiar la componente de alta energía de los estallidos. Los
% datos de Fermi han mostrado que algunos GRB emiten fotones \(\geq
% 100~\mathrm{GeV}\) con un ligero retardo respecto de la emisión de menor
% energía, lo que sugiere procesos adicionales durante la evolución del chorro
% relativista. Un caso emblemático fue GRB 130427A, observado por Fermi y
% \textit{Swift}, que produjo un fotón de \(\approx 95~\mathrm{GeV}\) y mantuvo
% emisión gamma durante casi 24 horas, desafiando los modelos teóricos
% tradicionales y ayudando a refinar las teorías sobre cómo los chorros de GRB
% transfieren energía a su entorno. \emph{(Nota: el apodo “BOAT, Brightest Of All
% Time” suele usarse para GRB 221009A, detectado en 2022.)}\\


% Además de \textit{Swift} y \textit{Fermi}, existe una amplia red internacional
% de detectores dedicados a los GRB. La ESA opera desde 2002 el observatorio
% INTEGRAL, que ha registrado decenas de estallidos y, junto con \textit{Fermi},
% fue clave en la observación del GRB 170817A en 2017. \citep{ESAheic1717b}
% Misiones como HETE-2 (NASA/Japón) y AGILE (Italia) aportaron de forma decisiva
% en la década de 2000, y hoy una nueva generación de satélites ---por ejemplo,
% SVOM (franco-chino), Insight-HXMT (China), GLAST, entre otros--- continúa la
% búsqueda con capacidades de localización mejoradas. En tierra, numerosos
% telescopios ópticos robotizados (como el conjunto ROTSE o la red LCOGT) y
% radiotelescopios se encargan de seguir los \textit{afterglows} tras las alertas.
% Gracias a este despliegue tecnológico, en la actualidad se detecta al menos un
% GRB al día en algún punto del universo observable, y es posible estudiarlos en
% prácticamente todas las bandas del espectro electromagnético.\\


% \begin{table} %%%% example table
% \begin{tabular}{l c c c} 
%  \hline
%  Source & RA (J2000) & DEC (J2000) & $V_{\rm sys}$ \\ 
%         & [h,m,s]    & [o,','']    & \kms          \\
%  \hline
%  NGC\,253 & 	00:47:33.120 & -25:17:17.59 & $235 \pm 1$ \\ 
%  M\,82 & 09:55:52.725, & +69:40:45.78 & $269 \pm 2$ 	 \\ 
%  \hline
% \end{tabular}
% \caption{Random table with galaxies coordinates and velocities, Number the tables consecutively in
% accordance with their appearance in the text and place any table notes below the table body. Please avoid using vertical rules and shading in table cells.
% }
% \label{Table1}
% \end{table}

\section{Modelos Espectrales Utilizados}
\label{sec:modelos_espectrales}

Para caracterizar la emisión de los destellos de rayos gamma, se emplearon diversos modelos espectrales, tanto fenomenológicos como de motivación física. A continuación se describen las formulaciones matemáticas de cada función utilizada en este trabajo.

\subsection{Función de Band}
La función de Band es un modelo fenomenológico ampliamente utilizado para describir el espectro de la emisión inmediata (prompt) de los GRBs. Se define como una ley de potencias doble, suavemente conectada, y está parametrizada por el pico de energía del espectro $\nu F_{\nu}$. Su forma matemática es:
\begin{equation}
F(E) = A
\begin{cases}
\left(\frac{E}{E_{\text{piv}}}\right)^\alpha e^{-(2+\alpha)E/E_{\text{peak}}},
& \text{si } E < \frac{(\alpha-\beta)E_{\text{peak}}}{2+\alpha} \\
\left(\frac{(\alpha-\beta)E_{\text{peak}}}{(2+\alpha)E_{\text{piv}}}\right)^{\alpha-\beta}
e^{\beta-\alpha}\left(\frac{E}{E_{\text{piv}}}\right)^\beta, & \text{en otro caso}
\end{cases}
\label{eq:band}
\end{equation}
Donde los parámetros del modelo son:
\begin{itemize}
    \item $A$: La amplitud, en unidades de $\text{ph} \cdot \text{s}^{-1} \cdot \text{cm}^{-2} \cdot \text{keV}^{-1}$.
    \item $E_{\text{peak}}$: La energía del pico del espectro $\nu F_{\nu}$, en keV.
    \item $\alpha$: El índice fotónico de bajas energías.
    \item $\beta$: El índice fotónico de altas energías.
    \item $E_{\text{piv}}$: La energía pivote, usualmente fijada en 100 keV.
\end{itemize}

\subsection{Cuerpo Negro (BlackBody)}
El modelo de cuerpo negro describe la emisión puramente térmica de un objeto en equilibrio. Su espectro está determinado únicamente por su temperatura y se expresa mediante la ley de Planck:
\begin{equation}
F(E) = A \frac{E^2}{e^{E/kT}-1}
\label{eq:blackbody}
\end{equation}
Sus parámetros son:
\begin{itemize}
    \item $A$: La amplitud o constante de normalización, en $\text{ph} \cdot \text{s}^{-1} \cdot \text{cm}^{-2} \cdot \text{keV}^{-1}$.
    \item $kT$: La temperatura del cuerpo negro, en unidades de keV.
\end{itemize}

\subsection{Modelo Comptonizado (Comptonized)}
Este modelo, también conocido como ley de potencias con un corte exponencial (\textit{cutoff power-law}), es a menudo utilizado para describir procesos de Comptonización, donde fotones de baja energía ganan energía a través de colisiones con una población de electrones calientes. La función se define como:
\begin{equation}
F(E) = A \left(\frac{E}{E_{\text{piv}}}\right)^\lambda e^{-(2+\lambda)E/E_{\text{peak}}}
\label{eq:comptonized}
\end{equation}
Donde los parámetros son:
\begin{itemize}
    \item $A$: La amplitud, en $\text{ph} \cdot \text{s}^{-1} \cdot \text{cm}^{-2} \cdot \text{keV}^{-1}$.
    \item $E_{\text{peak}}$: La energía del pico del espectro $\nu F_{\nu}$, que corresponde a la energía de corte o de plegado (\textit{e-folding energy}), en keV.
    \item $\lambda$: El índice fotónico de la ley de potencias.
    \item $E_{\text{piv}}$: La energía pivote, fijada en 100 keV.
\end{itemize}

\subsection{Ley de Potencias (PowerLaw)}
La ley de potencias es el modelo más simple para describir la radiación no térmica, como la emitida por procesos de sincrotrón. Su forma matemática es:
\begin{equation}
F(E) = A \left(\frac{E}{E_{\text{piv}}}\right)^{\lambda}
\label{eq:powerlaw}
\end{equation}
Sus parámetros son:
\begin{itemize}
    \item $A$: La amplitud en la energía pivote, en $\text{ph} \cdot \text{s}^{-1} \cdot \text{cm}^{-2} \cdot \text{keV}^{-1}$.
    \item $\lambda$: El índice fotónico, que describe la pendiente del espectro.
    \item $E_{\text{piv}}$: La energía pivote, en keV, en la cual se normaliza la amplitud. Generalmente se fija en 100 keV.
\end{itemize}

\section{Metodología}
\label{metodología}

% Nos basamos en el texto de \citep{ackermann2011}...

% Utilizamos el módulo gdt-fermi \citep{GDT-Fermi} de las Gamma-Ray Data Tools \citep{GDT-Core} para el análisis de los datos y la realización de un encaje espectral.

Se analizaron dos destellos de rayos
gamma, bn090926181 y bn090424592. Puesto que el
análisis final se hará sobre este segundo objeto, el proyecto se construyó
alrededor del primero ya que este ha sido estudiado previamente en
\cite{ackermann2011}.

Para el análisis de los destellos de rayos gamma (GRB), se utilizó el paquete de
software \texttt{Gamma-ray Data Tools} (GDT-Core) \citep{GDT-Core}. Se
procesaron los datos provenientes de los detectores de Ioduro de Sodio (NaI) 3,
6 y 7, y del detector de Germanato de Bismuto (BGO) 0, para el objeto
bn090926181. Dichos datos fueron obtenidos del catálogo de Triggers de Fermi GBM
\citep{von_Kienlin_2020} \citep{Gruber_2014} \citep{von_Kienlin_2014}
\citep{Bhat_2016}.

La primera etapa del análisis consistió en la generación de curvas de luz para
estudiar la evolución temporal del destello. Específicamente, se crearon tres
curvas de luz a partir de los datos del detector NaI 3, cubriendo las bandas de
energía de 8-50 keV, 50-300 keV y 300-900 keV. Estas se representaron
gráficamente sobre un eje temporal común para facilitar la comparación de la
emisión en los diferentes rangos.

Posteriormente, se procedió con el análisis espectral. Primero, se realizó un
ajuste polinomial al fondo (\textit{background}) en las curvas de luz para
establecer y sustraer el nivel de ruido base. A continuación, se seleccionó el
intervalo de tiempo correspondiente al pico de emisión del destello. Los datos
de este intervalo, junto con la matriz de respuesta instrumental (RSP2) de cada
detector, se ajustaron utilizando un modelo de función de Band. Este
procedimiento permite determinar los cinco parámetros del modelo: la amplitud
(A), la energía del pico del espectro ($E_{\text{peak}}$), el índice fotónico de
baja energía ($\alpha$), el índice fotónico de alta energía ($\beta$) y la
energía de pivote.

Para examinar la evolución temporal de los parámetros espectrales, se implementó un análisis resuelto en el tiempo. Se desarrolló un procedimiento iterativo que segmenta la curva de luz en intervalos de tiempo de duración configurable. Para cada uno de estos intervalos, se aplicó el modelo de ajuste de Band. En los casos donde el algoritmo de ajuste no lograba la convergencia, el análisis avanzaba al siguiente intervalo hasta obtener un resultado estadísticamente válido. Los valores de los parámetros obtenidos en cada ajuste, junto con sus incertidumbres, se tabularon y graficaron en función del tiempo para estudiar su dinámica.

Adicionalmente, se repitió este análisis espectral utilizando otros modelos teóricos para describir la emisión del destello. Específicamente, se aplicaron ajustes con una función de cuerpo negro (Blackbody), un modelo Comptonizado y una ley de potencias (PowerLaw). Para cada uno de los modelos evaluados, se generaron representaciones gráficas del espectro de energía, en la forma $\nu F_{\nu}$, superpuestas con el ajuste correspondiente para evaluar la calidad del mismo.

\section{Resultados y Discusión}
\label{resultados}

\begin{figure}
    \centering
    \includegraphics[width=0.6\linewidth]{imagenes/fit_bnbn090424592_band.pdf}
    \caption{Ajuste espectral para el intervalo de tiempo donde el conteo fue pico para el objeto bn090424592. \textbf{Panel superior:} Muestra el espectro de energía en unidades de tasa de cuentas (puntos con barras de error) para los detectores NaI (n7, n8, nb) y BGO (b1). La línea continua sobrepuesta a los datos representa el mejor ajuste obtenido con el modelo [Nombre del Modelo, e.g., Band].
        \textbf{Panel inferior:} Presenta los residuales del ajuste, calculados como la diferencia entre los datos observados y el modelo, en unidades de desviación estándar ($\sigma$). La distribución de los residuales en torno a cero indica que el modelo describe adecuadamente los datos en todo el rango de energía analizado.}
    \label{fig:fit_bnbn090424592_band}
\end{figure}

\begin{figure}
    \centering
    \includegraphics[width=0.5\linewidth]{imagenes/espectro_vFv_bnbn090424592_band.pdf}
    \includegraphics[width=0.5\linewidth]{imagenes/espectro_vFv_bnbn090424592_multiple.pdf}
    \caption{Visualización del modelo espectral ajustado en el formato $\nu F_{\nu}$. \textbf{(Superior)} El modelo de Band, que describe fenomenológicamente la emisión global del destello. El pico de esta curva representa la energía a la que se emite la mayor parte de la potencia ($E_{\text{peak}}$). \textbf{(Inferior)} Descomposición del espectro total en sus componentes físicos subyacentes: un cuerpo negro (\textit{BlackBody}), un modelo Comptonizado y una ley de potencias (\textit{PowerLaw}). Las líneas tenues para cada componente representan muestras de la distribución posterior de los parámetros, ilustrando la incertidumbre del ajuste. La suma de estos componentes (línea morada) reconstruye el espectro total observado.}
    \label{fig:placeholder}
\end{figure}


% Explicación de gráficas.

\begin{figure}
    \centering
    \includegraphics[width=\linewidth]{imagenes/grafica_band_181_2.0_conbgo.png}
    \caption{Evolución temporal de los parámetros para el ajuste de tipo Band del objeto bn090926181}
    \label{fig:181-band}
\end{figure}
\begin{figure}
    \centering
    \includegraphics[width=\linewidth]{imagenes/grafica_band_592_0.5.png}
    \caption{Evolución temporal de los parámetros para el ajuste de tipo Band del objeto bn090424592.}
    \label{fig:592-band}
\end{figure}

La Figura \ref{fig:181-band} muestra la evolución temporal de los parámetros de encaje de Band para el objeto bn090926181, los conteos por segundo
(gris) y los ajustes para los parámetros de Energía pico (azul), $\alpha$
(verde) y $\beta$ (rojo) del objeto bn090926181, y lo mismo vemos representado en la figura \ref{fig:592-band} para el encaje de Band para el objeto bn090424592. En ambos casos podemos observar que la energía y el parámetro alpha tienden a seguir la forma general de la curva de luz mientras que el parámetro beta tiene un comportamiento más errático, mostrando barras de error mucho más grandes cuando el tamaño del bin es muy pequeño, pero que se puede mejorar utilizando un tamaño de bin mayor al costo de hacer menos evidente la correlación entre los parámetros de energía y alpha con la curva de luz. Esto se puede observar al notar como beta tiene un mejor comportamiento en la figura \ref{fig:181-band} donde se utilizó un bin de $2$ s contra los $0.5$ s en todos los otros análisis.

\begin{figure}
    \centering
    \includegraphics[width=\linewidth]{imagenes/grafica_multi_181_0.5s.png}
    \caption{Evolución temporal de los parámetros para el ajuste de tipo Multicomponente del objeto bn090926181}
    \label{fig:181-multi}
\end{figure}

\begin{figure}
    \centering
    \includegraphics[width=\linewidth]{imagenes/grafica_multi_592_0.5s.png}
    \caption{Evolución temporal de los parámetros para el ajuste de tipo Multicomponente del objeto bn090926181}
    \label{fig:592-multi}
\end{figure}

Ambos eventos fueron graficados por un período de $19$ s, $0-18$. Pero el análisis temporal del espectro se muestra en el rango $(0.1,17)$ s para el objeto bn090926181 y rango $(0.1,7.0)$ s para el objeto bn090424592, dado que este último tiene una señal mucho más débil. Como podemos observar en las gráficas \ref{fig:592-band}, \ref{fig:592-multi} la señal deja de ser diferenciable del ruido de fondo al rededor de $5$ segundos después del evento.

En las figuras \ref{fig:181-multi} y \ref{fig:592-multi} representamos el modelo multicomponente, de nuevo mostrando la curva de luz de los eventos y luego la evolución espectral de los parámetros, en este caso, se puede observar que la energía del componente Comptomized y la temperatura del componente BlackBody se mueven reflejando la curva de luz del evento, mientras el índice fotónico del componente PowerLaw muestra un comportamiento con mayor incerteza.

% Mostrar las gráficas.


% Discusión de las gráficas.

%% The Appendices part is started with the command \appendix;
%% appendix sections are then done as normal sections
% \appendix

% \section{Appendix title 1}
%% \label{}

% \section{Appendix title 2}
%% \label{}

%% If you have bibdatabase file and want bibtex to generate the
%% bibitems, please use
%%
\bibliographystyle{elsarticle-harv} 
\bibliography{bibliography}

%% else use the following coding to input the bibitems directly in the
%% TeX file.

%%\begin{thebibliography}{00}

%% \bibitem[Author(year)]{label}
%% For example:

%% \bibitem[Aladro et al.(2015)]{Aladro15} Aladro, R., Martín, S., Riquelme, D., et al. 2015, \aas, 579, A101


%%\end{thebibliography}

\end{document}

\endinput
%%
%% End of file `elsarticle-template-harv.tex'.
