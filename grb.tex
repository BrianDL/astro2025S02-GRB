%% 
%% Copyright 2007-2020 Elsevier Ltd
%% 
%% This file is part of the 'Elsarticle Bundle'.
%% ---------------------------------------------
%% 
%% It may be distributed under the conditions of the LaTeX Project Public
%% License, either version 1.2 of this license or (at your option) any
%% later version.  The latest version of this license is in
%%    http://www.latex-project.org/lppl.txt
%% and version 1.2 or later is part of all distributions of LaTeX
%% version 1999/12/01 or later.
%% 
%% The list of all files belonging to the 'Elsarticle Bundle' is
%% given in the file `manifest.txt'.
%% 
%% Template article for Elsevier's document class `elsarticle'
%% with harvard style bibliographic references

%\documentclass[preprint,12pt,authoryear]{elsarticle}

%% Use the option review to obtain double line spacing
%% \documentclass[authoryear,preprint,review,12pt]{elsarticle}

%% Use the options 1p,twocolumn; 3p; 3p,twocolumn; 5p; or 5p,twocolumn
%% for a journal layout:
%% \documentclass[final,1p,times,authoryear]{elsarticle}
%% \documentclass[final,1p,times,twocolumn,authoryear]{elsarticle}
%% \documentclass[final,3p,times,authoryear]{elsarticle}
%% \documentclass[final,3p,times,twocolumn,authoryear]{elsarticle}
%% \documentclass[final,5p,times,authoryear]{elsarticle}
\documentclass[final,5p,times,twocolumn,authoryear]{elsarticle}

%--- idioma -----------------------------------------------------------
\usepackage[spanish]{babel}          % opcional: traducir los encabezados
\usepackage[utf8]{inputenc}
\usepackage[T1]{fontenc}

%--- redefinir los encabezados ----------------------------------------
\abstracttitle{Resumen}
\keywordtitle{Palabras clave}

%% For including figures, graphicx.sty has been loaded in
%% elsarticle.cls. If you prefer to use the old commands
%% please give \usepackage{epsfig}

%% The amssymb package provides various useful mathematical symbols
\usepackage{amssymb}
% \usepackage{lipsum}
%% The amsthm package provides extended theorem environments
%% \usepackage{amsthm}

%% The lineno packages adds line numbers. Start line numbering with
%% \begin{linenumbers}, end it with \end{linenumbers}. Or switch it on
%% for the whole article with \linenumbers.
%% \usepackage{lineno}

%% You might want to define your own abbreviated commands for common used terms, e.g.:
\newcommand{\kms}{km\,s$^{-1}$}
\newcommand{\msun}{$M_\odot}

\journal{Astronomy $\&$ Computing}


\begin{document}

\begin{frontmatter}

%% Title, authors and addresses

%% use the tnoteref command within \title for footnotes;
%% use the tnotetext command for theassociated footnote;
%% use the fnref command within \author or \affiliation for footnotes;
%% use the fntext command for theassociated footnote;
%% use the corref command within \author for corresponding author footnotes;
%% use the cortext command for theassociated footnote;
%% use the ead command for the email address,
%% and the form \ead[url] for the home page:
%% \title{Title\tnoteref{label1}}
%% \tnotetext[label1]{}
%% \author{Name\corref{cor1}\fnref{label2}}
%% \ead{email address}
%% \ead[url]{home page}
%% \fntext[label2]{}
%% \cortext[cor1]{}
%% \affiliation{organization={},
%%            addressline={}, 
%%            city={},
%%            postcode={}, 
%%            state={},
%%            country={}}
%% \fntext[label3]{}

\title{Title of paper}

%% use optional labels to link authors explicitly to addresses:
%% \author[label1,label2]{}
%% \affiliation[label1]{organization={},
%%             addressline={},
%%             city={},
%%             postcode={},
%%             state={},
%%             country={}}
%%
%% \affiliation[label2]{organization={},
%%             addressline={},
%%             city={},
%%             postcode={},
%%             state={},
%%             country={}}

\author[first]{Author name}
\affiliation[first]{organization={ECFM, Universidad de San Carlos de Guatemala},%Department and Organization
            addressline={}, 
            city={Guatemala},
            postcode={}, 
            state={Guatemla},
            country={Guatemala}}


\author[second]{Brian Leiva}
\affiliation[second]{organization={ECFM, Universidad de San Carlos de Guatemala},%Department and Organization
            % city={Guatemala},
            % state={Guatemala},
            country={Guatemala}}

\author[third]{Alessandro Lavagnino}
\affiliation[third]{organization={ECFM, USAC},%Department and Organization
            % addressline={}, 
            city={Guatemala},
            % postcode={}, 
            % state={Guatemala},
            country={Guatemala}}

\begin{abstract}
%% Text of abstract
Example abstract for the astronomy and computing journal. Here you provide a brief summary of the research and the results.
\end{abstract}

%%Graphical abstract
%\begin{graphicalabstract}
%\includegraphics{grabs}
%\end{graphicalabstract}

%%Research highlights
%\begin{highlights}
%\item Research highlight 1
%\item Research highlight 2
%\end{highlights}

\begin{keyword}
%% keywords here, in the form: keyword \sep keyword, up to a maximum of 6 keywords
estallido de rayos gamma: individual: GRB 090926A \sep
Astrofísica – Fenómenos astrofísicos de alta energía \sep
Astrofísica – Cosmología y astrofísica extragaláctica



%% PACS codes here, in the form: \PACS code \sep code

%% MSC codes here, in the form: \MSC code \sep code
%% or \MSC[2008] code \sep code (2000 is the default)

\end{keyword}


\end{frontmatter}

%\tableofcontents

%% \linenumbers

%% main text

\section{Introducción}
\label{introduction}

Los destellos de rayos gamma (GRB) son explosiones transientes de muy alta energía que liberan, en cuestión de segundos, una cantidad de energía comparable a la de una galaxia entera \citep{kouveliotou1993,piran1999}.  Su emisión se divide en dos fases observacionales.

En primer lugar, la emisión principal (prompt) constituye la fase inicial, cuya duración varía desde milisegundos hasta varios cientos de segundos.  Durante este intervalo se detecta radiación de alta energía (keV–GeV) caracterizada por flujos de fotones extremadamente rápidos y variaciones temporales muy rápidas \citep{gehrels2009,ackermann2009}.

Posteriormente, la emisión tardía (afterglow) se manifiesta cuando el choque relativista interactúa con el medio circundante, produciendo una radiación más prolongada que se extiende desde los rayos X hasta el rango radiofónico.  El afterglow puede persistir desde horas hasta varios meses, ofreciendo una valiosa información sobre los procesos de aceleración de partículas y la naturaleza del entorno del progenitor \citep{vanparadijs2000,sari1998}.

Comprender este doble comportamiento es esencial para identificar los mecanismos de producción de GRB, que pueden originarse tanto por el colapso de estrellas masivas como por la fusión de objetos compactos \citep{meszaros2006}.

\section{Antecedentes}
\label{antecedentes}
Nos basamos en el texto de \cite{ackermann2009}...

% \begin{table} %%%% example table
% \begin{tabular}{l c c c} 
%  \hline
%  Source & RA (J2000) & DEC (J2000) & $V_{\rm sys}$ \\ 
%         & [h,m,s]    & [o,','']    & \kms          \\
%  \hline
%  NGC\,253 & 	00:47:33.120 & -25:17:17.59 & $235 \pm 1$ \\ 
%  M\,82 & 09:55:52.725, & +69:40:45.78 & $269 \pm 2$ 	 \\ 
%  \hline
% \end{tabular}
% \caption{Random table with galaxies coordinates and velocities, Number the tables consecutively in
% accordance with their appearance in the text and place any table notes below the table body. Please avoid using vertical rules and shading in table cells.
% }
% \label{Table1}
% \end{table}


\section{Metodología}
\label{metodología}


\section{Resultados y Discusión}
\label{resultados}

%% The Appendices part is started with the command \appendix;
%% appendix sections are then done as normal sections
% \appendix

% \section{Appendix title 1}
%% \label{}

% \section{Appendix title 2}
%% \label{}

%% If you have bibdatabase file and want bibtex to generate the
%% bibitems, please use
%%
\bibliographystyle{elsarticle-harv} 
\bibliography{bibliography}

%% else use the following coding to input the bibitems directly in the
%% TeX file.

%%\begin{thebibliography}{00}

%% \bibitem[Author(year)]{label}
%% For example:

%% \bibitem[Aladro et al.(2015)]{Aladro15} Aladro, R., Martín, S., Riquelme, D., et al. 2015, \aas, 579, A101


%%\end{thebibliography}

\end{document}

\endinput
%%
%% End of file `elsarticle-template-harv.tex'.
